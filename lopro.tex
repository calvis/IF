A logic program is a set of initial facts (typically general and
hypothetical) from which new facts can be generated and queries on the
space of facts can be made. It can be viewed as an inference system and its
execution as proof search.

As a small example, consider the following Prolog (XXX cite) program to
find paths between nodes in a graph:

\begin{verbatim}
path X X.
path X Z
  :- edge Z Y, path Y Z.
\end{verbatim}



Here's a very minimal example of an interaction in linear logic
programming.

% XXX toggle


- Intensional vs extensional definition of a game object

- Logic program execution is proof search

- A game as a logic program: interactive proof search

