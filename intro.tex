Interactive fiction (hereafter IF), more commonly known as text adventures,
is a genre of game in which players interact in a text loop not unlike a
REPL. The game provides a prompt, the player enters a command, and the game
responds with the next prompt, possibly changing some internal state.

Typically, the game's state includes a map of a physical space that the
player navigates with directional commands. The space consists of connected
rooms, and within the rooms are objects and characters that the player can
interact with, e.g. by {\em looking} around, {\em examining} something, 
{\em taking} something, or {\em talking} to someone. The player also
typically has an inventory to store taken objects and can apply commands to
those objects.

The author of a piece of interactive fiction is tasked with describing a
rich setting and anticipating the player's actions.  With each new
interactable object introduced, the space of game play explodes
combinatorially. Of course, in order for the game to feel interactive, we
can't anticipate and script a response to every action; we need to set up
simple rules that {\em emergently generate} game content.  (In the game
design world, this notion of generativity is refered to as {\em
procedural}.) Most frameworks for writing interactive fiction deal with the
combinatorial explosion by establishing broad defaults for every command;
the IF author need only override default game responses to selected actions
upon a new object she introduces.

A key challenge in language design for any kind of game programming is
enabling such a rich space of possible games that feel generative and
interactive rather than canned.  IF in particular is a ripe domain for
programming languages research because it introduces the richness of game
design without the extra cruft of rendering. Everything is a turn-based,
discrete state transition. We can imagine turning this crank on an
extralinguistic interpreter; the language, then, need only be for {\em
specifying game logic}. A programming language for interactive fiction could
easily extend to prototyping games with fancier rendering systems, and
indeed some like-minded game programmers have done this:
http://eis-blog.ucsc.edu/2009/06/you-have-to-mine-the-ore/ XXX move this to
citation

The majority of IF has been developed in the systems Inform (XXX cite), TADS,
and ADRIFT. The authors of this work are primarily familiar with Inform,
specifically Inform7. In Inform7, the author specifies game behavior by
matching on a player action (possibly guarded by some condition) and
specifying a state change. We hope to base our design on this basic idea
but using ideas from logic programming and substructural logics to make
games easier to specify and reason about.

